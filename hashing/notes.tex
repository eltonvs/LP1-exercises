\documentclass[12pt,openright,oneside,a4paper,english,brazil]{abntex2}

\usepackage[brazil]{babel}
\usepackage[utf8]{inputenc}
\usepackage{indentfirst}
\usepackage{amssymb}

\autor{Elton de Souza Vieira}
\titulo{Relatório-Resumo sobre Tabelas de Dispersão}
\data{2016}
\local{Natal - RN}
\tipotrabalho{relatorio}

\DoubleSpacing
\setlength{\parindent}{1cm}

\begin{document}
\imprimircapa

\section*{Introdução}
    Basicamente, uma Tabela de Dispersão é uma estrutura de dados do tipo dicionário, ou seja, associa uma chave a um valor. Ela não armazena elementos repetidos, nem estabelece uma ordem entre eles (de forma que possa acessá-los sequencialmente) e, por causa disso, ela não consegue recuperar o sucessor ou antecessor a um elemento.

\section*{Hashing}
    Quando trabalhamos com estruturas do tipo chave $\rightarrow$ valor, geralmente temos um vetor (hash table) com \textit{n} posições (se as chaves forem números, \textit{n} seria a o valor da maior chave encontrada), dessa forma, poderíamos acessar um valor como vetor[chave].

    O ponto negativo de se fazer dessa forma é quando a chave não é um número (uma string com nomes, por exemplo) ou quando o número de chaves é pequeno, mas representado por um número grande (número de matrículas que seguem um padrão numérico), que fazem com que haja uma alocação de memória desnecessária. Para solucionar esse problema, foi criada a técnica de \textbf{hashing}.

\section*{Complexidade}

\section*{Relações com outras Estruturas de Dados}

\end{document}