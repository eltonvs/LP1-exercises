\documentclass[12pt,openright,oneside,a4paper,english,brazil]{abntex2}

\usepackage[brazil]{babel}
\usepackage[none]{hyphenat}
\usepackage[utf8]{inputenc}
\usepackage{indentfirst}
\usepackage{listings}
\usepackage{xcolor}

% Creates \code tag to print inline code
\definecolor{codegray}{gray}{0.9}
\newcommand{\code}[1]{\colorbox{codegray}{\texttt{#1}}}

% Document Info
\author{Elton de Souza Vieira\thanks{eltonviana@ufrn.edu.br || eltonviana.com}}
\title{Relatório-Resumo sobre Tabelas de Dispersão}
\date{2016}
\local{Natal - RN}
\tipotrabalho{relatorio}

% PDF Metadata
\hypersetup{
    pdfinfo={
        Title={\thetitle},
        Author={\theauthor}
    }
}

% Visual configuration
\OnehalfSpacing
\setlength{\parindent}{1cm}
\setlength{\parskip}{\baselineskip}

\begin{document}
\imprimircapa

\section*{Introdução}
    Basicamente, uma Tabela de Dispersão é uma estrutura de dados do tipo dicionário, ou seja, associa uma chave a um valor.
    Um elemento dessa estrutura é acessado pela sua chave (\textit{key}), o conjunto de todas as chaves formam o chamado \textit{universo de chaves}.

    Uma Tabela de Dispersão não armazena elementos repetidos, nem estabelece uma ordem entre eles (de forma que possa acessá-los sequencialmente) e, por causa disso, ela não consegue recuperar o sucessor ou antecessor a um elemento.

\section*{Hashing}
    Ao se trabalhar com estruturas do tipo dicionário (${chave}\rightarrow{valor}$), da forma mais simples, tem-se um vetor de registros (\textit{hash table}) com \textit{n-1} posições (no qual \textit{n} representa o tamanho do \textit{universo de chaves}) em que cada registro armazena uma estrutura que possui os campos de chave e valor.

    Conforme vai sendo utilizada, a tabela hash vai ficando ``bagunçada'', com espaços vazios entre os elementos. Assim, para realizar uma inserção é necessário encontrar um espaço que esteja vazio.
    Portanto, para encontrar a posição em que um elemento deve ser inserido em uma Tabela de Dispersão, foi criada a técnica de \textbf{hashing}.

    Essa técnica consiste na criação de uma função (\textit{Hash Function}) que, dado uma chave \textit{k}, retorne o \textbf{código de espelhamento} (\textit{hash code}) \textit{h} pertencente ao intervalo [0,~\textit{n}-1].
    É crucial que essa função retorne sempre o mesmo \textit{hash code} sempre que tiver a mesma chave passada por parâmetro.
    Uma forma simples de implementar essa função é com a aritmética modular, sendo \textit{k} a chave e \textit{n} o tamanho do vetor de registros, \code{$\lambda$~k,~n:~k~mod~n} (desde que \textit{k} seja um número inteiro).

    Como o tamanho do vetor de registros é limitado, é inevitável que a função retorne o mesmo código de espelhamento para duas entradas diferentes.
    Quando isso ocorre, dizemos que houve uma \textit{colisão}.
    Logo, para encontrar a posição que o elemento pode ser inserido no vetor é necessário percorrê-lo linearmente até que encontre uma posição vazia (aumentando a complexidade e, consequentemente, o tempo de execução).

    Uma função de espalhamento é considerada boa se e possui uma taxa de colisão \textbf{reduzida} e consegue distribuir \textbf{uniformemente} os elementos pelo vetor.
    Para desenvolver uma função dessa natureza é necessário analisar antes como as chaves são formadas, para então definir qual a função que melhor se comporte com o conjunto de entradas possíveis.

    Dessa forma, dada uma função de espelhamento F e dois conjuntos de entradas (A e B), é possível que F seja considerada boa para A, e ruim para B.
    Dependendo da função, há também como variar a eficiência da função apenas modificando o tamanho da Tabela Hash.

\section*{Complexidade}

\section*{Relações com outras Estruturas de Dados}

\section*{SHA-1}

\end{document}
